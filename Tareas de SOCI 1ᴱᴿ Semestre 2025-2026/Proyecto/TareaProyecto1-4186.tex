\documentclass[11pt]{article}
\usepackage{setspace}    
\onehalfspacing
\usepackage{graphicx}    
\usepackage{amsmath}     
%\usepackage{natbib}      % Apagado en esta tarea
\usepackage{array}       
\usepackage{multirow}    
\usepackage{siunitx}     
\usepackage{textcomp}    
\usepackage{inputenc} 
\usepackage[spanish,es-tabla]{babel}  
\usepackage{csquotes}   
\usepackage[top=1in,right=1in,left=1in,bottom=1in]{geometry} 
\usepackage{makecell}
\usepackage[nospace,spanish]{varioref}
\usepackage{hyperref}      

\hypersetup{
    unicode=true,            
    pdftoolbar=true,         
    pdfmenubar=true,         
    pdffitwindow=false,      
    pdfstartview={FitH},     
    pdftitle={},             
    pdfauthor={},            
    pdfsubject={},           
    pdfcreator={},           
    pdfproducer={},          
    pdfkeywords={},          
    pdfnewwindow=true,       
    colorlinks=true,         
    linkcolor=blue,          
    citecolor=blue,          
    filecolor=blue,          
    urlcolor=blue            
}

\usepackage[backend=biber,style=chicago-authordate,natbib=true,language=spanish]{biblatex} %paquete para manejar citas
\addbibresource{Referencias.bib} %Documento que contiene anotaciones bibliográficas

\title{SOCI 4186-002\\ Tareas de Proyecto de Investigación --\textnumero 1 \\ Pregunta inicial de investigación}
\author{Pon tu nombre aquí \\ Su número de estudiante}
\date{\today}

\begin{document}
\singlespacing
\maketitle
\onehalfspacing
\begin{center}
    Fecha límite de entrega: \underline{viernes, \textbf{11} de septiembre de 2025, a más tardar a las 23:59.}
\end{center}


\begin{enumerate}
    \item Descargarán este documento en el GitHub de la clase. Colóquenlo en .
    \item Cambiarán el nombre que sale arriba (al final del preámbulo, poco antes del \texttt{\textbackslash begin\{document\}}, donde dice \texttt{\textbackslash author\{Pon tu nombre aquí\}} y pondrán en su lugar su nombre(s) y apellidos.
    \item Prestarán atención a la ortografía de la lengua española, asegurándose de colocar acentos, diéresis, o tildes donde deban ir. Para esto, tienen dos opciones 
    \begin{enumerate}
        \item si su teclado estuviera en español, usen los símbolos donde es menester,
        \item si su teclado no estuviera en español, está la opción indicada en clase de usar \texttt{\textbackslash} y el símbolo necesario (por ejemplo, \textbackslash' o \textbackslash'{} para acento (\'a o \'{a}), \textbackslash\"\ o \textbackslash\"{} para diéresis (\"u o \"{u}), \textbackslash\textasciitilde\ para \~n)
        \item Alternativamente, si prefiriera escribir en inglés \underline{la tarea entera}, podrá hacerlo, ciñéndose estrictamente a una de las tres ortografías principales de ese idioma, es decir, inglés británico, inglés Oxford, o inglés estadounidense.
    \end{enumerate}
    \item Completarán como mejor sea posible la tarea, y podrán, si entienden necesario hacerlo, trabajar la tarea con compañeres de clase. Si así lo hicieren, pondrán antes de la sección 1 una oración indicándome con quién(es) trabajó la tarea. De lo contrario, dejarán la línea que está actualmente escrita.
    \item  Una vez terminen, apretarán el símbolo de descarga (a la derecha del botón de Compilar, con flecha descendiente), y descargarán el PDF para enviarlo a mi persona.
\end{enumerate}

\newpage

\section*{Pregunta Inicial de Investigación y Modalidad}
Les estudiantes identificarán un tema de interés, realizarán una breve búsqueda sobre su tema para que puedan empezar a concretar de tema de interés a pregunta. Deberán entregar un documento de al menos 300 palabras que incluya los siguientes componentes (por favor, etiquete cada componente como una subsección):
\begin{enumerate}
    \item Su tema de investigación principal.
    \item Una pregunta de investigación tentativa, o varias preguntas si no estuvieran segures de qué dirección tomar.
    \item Una breve explicación de por qué desean explorar la pregunta/tema y por qué merece un análisis académico cuantitativo.
    \item Indicación de modalidad de estudio: grupal o individual. Si fuera grupal, información de miembros que conformen el grupo y nombre para el grupo. De una vez, indique cuál de las tres modalidades de investigación usará.
\end{enumerate}

\section{Ejemplo de respuesta}

\subsection{Tema y pregunta}
Mi tema es \textbf{capitalismo comparado}, en especial las variedades de capitalismo seg\'un \citet{hall2001introduction}. Espec\'ificamente, la discusi\'on que se elaborara en a\~nos posteriores \citep{hall2009institutional}, me parece una direcci\'on \'util que me lleva a la pregunta, \textit{podría medirse cuantitativamente el concepto de coordinaci\'on industrial para comparar países a través del tiempo?}.

\subsection{Justificación}
Según \citet{giovannetti2015}, tal cosa es importante. Sin embargo, esto va contra lo señalado por \citet{lopezroman2012} quien indica que mengana cosa es lo verdaderamente importante. En esto entendemos lo 1"o es realmente crucial, lo que señalaran las poetas en su momento, que \begin{quoting}y aserejé-ja-dejé. De jebe tu de jebere seibiunouva majavi an de bugui an de güididípi\end{quoting}, es decir, \textit{I said a hip hop the hip. The hippie to the hip hip hoppa ya don’t stop. The rockin’ to the bang-bang boogie said up jumps the boogie} en el cierre y albores de los siglos \lsc{XX} y \lsc{XXI} \citep{Hails2022}. Sin embargo, otros indican que esto no es adecuado \citep{nunn2011slave}.

Se ha discutido ampliamente este tema por varios acad\'emicos \citep{giovannetti2015,hall2009institutional} tal cosa...

\subsection{Modalidad}
Este estudio, de modalidad comparativa/explicativa, estará siendo de carácter individual, como puede verse en \vref{tabla:miembros_grupo}. 

\section*{Miembros del Grupo}
Si el trabajo se realiza en grupo, los miembros del grupo deberán ser declarados en la tabla siguiente. Denle un nombre al grupo en este caso. El límite es de tres miembros. De no ser en grupo, meramente declarará el estudiante que estará trabajando en esto por su parte.

\begin{table}[h!]
    \centering
    \begin{tabular}{|c|c|c|}
        \hline
        \textbf{Nombre de Estudiante} & \textbf{Concentración} & \textbf{\textnumero\ de Estudiante}\\ \hline
        Miembro 1 & &\\ \hline
        Miembro 2 & &\\ \hline
        Miembro 3 & &\\ \hline
    \end{tabular}
    \caption{Miembros del grupo `los Pitirres'}
    \label{tabla:miembros_grupo}
\end{table}

\newpage
\printbibliography[title={Bibliografía}, heading=subbibliography]

\end{document}