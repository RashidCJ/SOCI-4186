\documentclass[11pt]{article}
\usepackage{setspace}    
\onehalfspacing
\usepackage{graphicx}    
\usepackage{amsmath}     
%\usepackage{natbib}      % Apagado en esta tarea
\usepackage{array}       
\usepackage{multirow}    
\usepackage{siunitx}     
\usepackage{textcomp}    
%\usepackage[utf8]{inputenc}  %ignorado, el compilador ya tiene utf8
\usepackage[spanish,es-tabla]{babel}  
\usepackage{csquotes}        
\usepackage[top=1in,right=1in,left=1in,bottom=1in]{geometry} 
\usepackage{makecell}
\usepackage{hyperref}        

\hypersetup{
    unicode=true,            
    pdftoolbar=true,         
    pdfmenubar=true,         
    pdffitwindow=false,      
    pdfstartview={FitH},     
    pdftitle={},             
    pdfauthor={},            
    pdfsubject={},           
    pdfcreator={},           
    pdfproducer={},          
    pdfkeywords={},          
    pdfnewwindow=true,       
    colorlinks=true,         
    linkcolor=blue,          
    citecolor=blue,          
    filecolor=blue,          
    urlcolor=blue            
}

\usepackage[backend=biber,style=apa,natbib=true,language=spanish]{biblatex}
\addbibresource{Referencias.bib}

\setlength\bibitemsep{1.5\baselineskip}
\DeclareBibliographyDriver{newspaper}{
  \usebibmacro{author}%
  \newunit\newblock
  % Fecha en español: (dd de mm de aaaa)
  \printtext[parens]{%
    \iffieldsequal{year}{}
      {}
      {%
        \printfield[day]{day}\space
        de\space
        \printfield[month]{month}\space
        de\space
        \printfield{year}%
      }%
  }%
  \newunit\newblock
  % Título
  \printfield{title}\addperiod\newunit\newblock
  % Periódico en cursiva
  \printfield[journaltitle]{journaltitle}\addperiod\newunit\newblock
  % URL
  \printfield{url}%
  \finentry
}

\DeclareLanguageMapping{spanish}{spanish-apa}

% Personalización para agregar bullets a la bibliografía
\defbibenvironment{bibliography}
  {\list{\labelitemi} % Usa el símbolo de bullet por defecto
    {\setlength{\leftmargin}{0pt}%
     \setlength{\itemindent}{\labelwidth}%
     \addtolength{\itemindent}{\labelsep}%
     \setlength{\itemsep}{\bibitemsep}%
     \setlength{\parsep}{\bibparsep}}}
  {\endlist}
  {\item}

\title{SOCI 4186-002\\ Tarea de Proyecto de Investigación \textnumero 3 \\ Borrador de la revisión de la literatura \\ \textsc{Individual o Nombre de Grupo}}
\author{Pon tu nombre aquí \\ Número de estudiante}
\date{\today} %dejen esto como está, con \today

\begin{document}
\singlespacing
\maketitle
Fecha de entrega: viernes 24 de octubre de 2025.

\begin{enumerate}
    \item Descargarán este documento en el GitHub de la clase y lo subirán a la una carpeta en Overleaf. El compilador a usar es \texttt{XeLaTeX}.
    \item Si su investigación se desarrolla de modo individual, dejen \textsc{Individual}; si es grupal, cambien el texto de plantilla por el nombre del grupo.
    \item Cambien el campo \texttt{\textbackslash author\{Pon tu nombre aquí\}}. Si es individual: su nombre completo seguido de \textbackslash\textbackslash y su número de estudiante. Si es en grupo, usen el formato: \texttt{\textbackslash author\{Estudiante 1 \textbackslash\textbackslash \ \textnumero\ de Estudiante 1 \textbackslash and Estudiante 2 \textbackslash\textbackslash \ \textnumero\ de Estudiante 2\textbackslash and Estudiante 3 \textbackslash\textbackslash \ \textnumero\ de Estudiante 3 \textbackslash and Estudiante 4 \textbackslash\textbackslash\ \textnumero\ de Estudiante 4\}}.
    \item Presten atención a la ortografía y sintaxis española (acentos, tildes, diéresis) o escriban en una de las tres modalidades de inglés discutidas: británico, Oxford o estadounidense. Cíñanse a una lengua a través de todo el documento.
%    \item Completarán como mejor sea posible la tarea, y podrán, si entienden necesario hacerlo, trabajar la tarea con compañeres de clase. Si así lo hicieren, pondrán antes de la sección 1 una oración indicándome con quién(es) trabajó la tarea. De lo contrario, dejarán la línea que está actualmente escrita.
    \item Descarguen el PDF tras compilar (icono de flecha descendente) y lo someterán a Teams.
\end{enumerate}

\section*{Formato del documento}
Siga esta plantilla, borrando las instrucciones si así deseare tras concluir las respuestas o tener el PDF de instrucciones a la mano. El documento debe incluir:
\begin{itemize}
    \item Revisión de literatura.
    \item Bibliografía en formato APA.
\end{itemize}
\onehalfspacing
\section*{Revisión de literatura}
Elaboración de un primer borrador de la revisión de la literatura basada en la bibliografía recopilada. Use adecuadamente citas textuales como señalaba a modo de ejemplo el autor \citet{hall2001introduction}, y parentéticas \citep[e.g.,][, entre otros]{hall2009institutional}, y cree una narrativa que atienda los puntos principales que quiere abordar. Y entonces \citet{figueroa2025captado} da esta evidencia.

\printbibliography[title={Bibliografía}, heading=subbibliography]

\end{document}